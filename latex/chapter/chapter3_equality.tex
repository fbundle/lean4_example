\chapter{EQUALITY: CALCULATION PROOF}


\section{EQUALITY}


Let \hl{a} and \hl{b} be any objects, then \hl{a = b} is a proposition (also written as \hl{Eq a b}). \hl{Eq} admits several properties

\begin{itemize}
	\item (reflexive): \hl{Eq.refl}
	
	Let \hl{a} be any object, then \hl{Eq.refl a: a = a} is a proof
	
	\item (symmetric): \hl{Eq.symm}
	
	Let \hl{hab: a = b} be a proof, then \hl{Eq.symm hab: b = a} is also a proof
	
	\item (transitive): \hl{Eq.trans}
	
	Let \hl{hab: a = b}, \hl{hbc: b = c} be proofs, then \hl{Eq.trans hab hbc: a = c} is also a proof
\end{itemize}

In proving equality, lean provides several substitution rules

\begin{itemize}
	\item \hl{Eq.subst}
	
	Let \hl{x: $\alpha$} and \hl{y: $\alpha$}. Let \hl{h: a = b} be a proof and \hl{p: $\alpha \to$ Prop}, then if \hl{hx: p x} is a proof, then \hl{Eq:subst: p y} is also a proof
	
	\item \hl{congrArg}
	
	Let \hl{x: $\alpha$} and \hl{y: $\alpha$}. Let \hl{h: a = b} be a proof and \hl{f: $\alpha \to \beta$}, then \hl{congrArg f h: f x = f y} is a proof
	
	\item \hl{congrFun}
	
	Let \hl{f: $\alpha \to \beta$} and \hl{g: $\alpha \to \beta$}. Let \hl{h: f = g} and \hl{x: $\alpha$}, then \hl{congrFun h x: f x = g x} is a proof
	
	\item \hl{congr}
	
	Let \hl{x: $\alpha$} and \hl{y: $\alpha$}. Let \hl{f: $\alpha \to \beta$} and \hl{g: $\alpha \to \beta$}. Let \hl{h\_1: x = y} and \hl{h\_2: f = g}, then \hl{congr h\_2 h\_1: f x = g y} is a proof
\end{itemize}

\section{CALCULATION PROOF - REWRITE TACTIC}

See examples for \hl{calc} and \hl{rewrite}, \hl{simp} tactics in \href{theorem_proving_in_lean_4/theorem_proving/calculation_proof.lean}{calculation\_proof.lean}

\section{TRANSITIVITY AND CALCULATION PROOF}

Calculation proof also works for inequalities or more generally any transitive binary relation

See examples in \href{theorem_proving_in_lean_4/theorem_proving/calculation_proof_for_transitivity.lean}{calculation\_proof\_for\_transitivity.lean}

\note{NOTE: we might not understand everything at the moment - bare with me, accept it exists and move on first}