\chapter{PREDICATE LOGIC}

Keywords: \hl{Prop}, CH, \textit{witnessed}, \textit{proof}, \textit{inhabited}, \textit{uninhabited}, \textit{introduction rule}, \textit{elimination rule}

\hl{Prop} is an object at level 1 which is the same level with \hl{Nat}. An element (level 0) of universe \hl{Prop} is called a proposition. For example, \hl{True}, \hl{1 + 1 = 3}, \hl{Fermat's last theorem}.

Curry–Howard (CH) \cite{CH} asserts that proving a proposition \hl{p: Prop} is equivalent to constructing an element \hl{x: p} at level -1. Hence, a proposition \hl{p: Prop} is said to be true if and only if there exists an element  \hl{x: p} at level -1, we say that the truth of \hl{p: Prop} is \textit{witnessed} by \textit{proof} \hl{x: p}. We say \hl{p: Prop} is \textit{inhabited} if there is a proof for \hl{p: Prop}, \hl{p: Prop} is \textit{uninhabited} if there is no proof for \hl{p: Prop}

The type system in lean is powerful enough to model predicate logic, we will describe below the builtin logical connectives and related \textit{introduction rules} and \textit{elimination rules}. Given a proposition \hl{p: Prop}, an introduction rule is a way to prove \hl{p: Prop} from something else, an elimination rule is a way deduce from \hl{p: Prop}

\section{TRUE}

\hl{True: Prop} is a proposition in lean.

\hl{True} admits an introduction rule \hl{True.intro: True} (also written as \hl{trivial: True})

\hl{True} admits no elimination rule, that is, one cannot use \hl{True} to prove anything

\section{FALSE}

\hl{False: Prop} is a proposition in lean.

\hl{False} admits no introduction rule, that is one cannot prove \hl{False}

\hl{False} admits an elimination rule \hl{False.elim: $\forall$ (q : Prop), False $\to$ q}, that is, one can prove everything from \hl{False}, given any proposition \hl{q: Prop}, \hl{False.elim q} is a proof for \hl{q: Prop}

\section{IMPLICATION $\to$}

Given two propositions \hl{p: Prop} and \hl{q: Prop}, \hl{f: p $\to$ q} is a map from type \hl{p} into type \hl{q}. If \hl{hp: p} is a proof for \hl{p}, then one obtain a proof for \hl{q} by \hl{f hp: q}. This is known as \textit{modus ponens} or implication elimination rule

\hl{p $\to$ q: Prop} is a proposition in lean.

\hl{p $\to$ q} admits implication elimination rule as described above and it admits introduction rule by declaring function of type \hl{p $\to$ q}

\section{CONJUNCTION $\land$}

Given two propositions \hl{p: Prop} and \hl{q: Prop}, \hl{p $\land$ q} is a proposition (also written as \hl{And p q} or \hl{p $\times$ q})

\hl{And} admits an and-introduction rule \hl{And.intro} as follows: given a proof \hl{hp: p} for \hl{p} and a proof \hl{hq: q} for \hl{q}, one can obtain a proof for \hl{p $\land$ q} by \hl{And.intro hp hq: p $\land$ q}

\hl{And} admits two and-elimination rules as follows: given a proof \hl{hpq: p $\land$ q} for \hl{p $\land$ q}, one can obtain a proof for \hl{p} by \hl{And.left hpq: p} and a proof for \hl{q} by \hl{And.right hpq: q} (also written as \hl{hpq.left} and \hl{hpq.right})


\section{DISJUNCTION $\lor$}

Given two propositions \hl{p: Prop} and \hl{q: Prop}, \hl{p $\lor$ q} is a proposition (also written as \hl{Or p q} or \hl{p $\oplus$ q})

\hl{Or} admits two or-introduction rules as follows: given a proof \hl{hp: p} for \hl{p}, one can obtain a proof for \hl{p $\lor$ q} by \hl{Or.intro\_left q hp}, given a proof \hl{hq: q} for \hl{q}, one can obtain a proof for \hl{p $\lor$ q} by \hl{Or.intro\_right p hq} (also written as \hl{Or.inl hp} and \hl{Or.inr hq})

\hl{Or} admits an or-elimination rule as follows: given a proof \hl{hpq: p $\lor$ q} for \hl{p $\lor$ q}, a proof \hl{hpr: p $\to$ r} for \hl{p $\to$ r}, and a proof \hl{hqr: q $\to$ r} for \hl{q $\to$ r}, one can obtain a proof for \hl{r} by \hl{Or.elim hpq hpr hqr}

\section{NEGATION $\lnot$}

Given a proposition \hl{p: Prop}, \hl{$\lnot$ p} is a propositon (also written as \hl{Not p}). \hl{$\lnot$ p} is precisely \hl{p $\to$ False}

\section{LOGICAL EQUIVALENCE $\leftrightarrow$}

Given two propositions \hl{p: Prop} and \hl{q: Prop}, \hl{p $\leftrightarrow$ q} is a proposition (also written as \hl{Iff p q})

\hl{Iff} admits an iff-introduction rule as follows: given a proof \hl{hpq: p $\to$ q} and a proof \hl{hqp: q $\to$ p}, one can obtain a proof for \hl{p $\leftrightarrow$ q} by \hl{Iff.intro hpq hqp}

\hl{Iff} admits two iff-elimination rules as follows: given a proof \hl{h: p $\leftrightarrow$ q}, one can obtain a proof for \hl{p $\to$ q} by \hl{Iff.mp h} and a proof for \hl{q $\to$ p} by \hl{Iff.mpr h}


\section{UNIVERSAL QUANTIFIER $\forall$}

\section{EXISTENTIAL QUANTIFIER $\exists$}

